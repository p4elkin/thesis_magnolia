\chapter{Project Implementation}
\label{implementation}
In Project Implementation chapter we will provide some specific details of the
system was built. We will touch the data binding problem, the gesture
recognition solution and some approaches to increase the client-side
performance. In addition to that we will give an example of a Magnolia 5.0 App
built on top of the aforementioned APIs.

\section{Server Side}

\section{Dialogs and Forms}
\input{include/architecture/forms_and_dialogs.tex}

\subsection{Workbench}

\subsection{Managing Multiple Browser Tabs With Vaadin 6}

\subsection{Wiring JCR with Vaadin Container API}
 
\subsection{JcrNodeAdapter}

\subsection{AbstractJcrContainer}

\subsection{Server Push}

\subsubsection{Principles of IcePush component}

\subsubsection{Application of IcePush in MagnoliaShell}

\section{Client Side}

\subsection{Fast Transitions with JQueryWrapper and CSS3 Delegate Plugin}

\subsubsection{Plugin Architecture Examination}

\subsubsection{Implementation}

\subsection{Adding Touch Devices Support with MGWT Library. Gesture Recognition}

\subsubsection{Why use mgwt?}

\subsubsection{Conflicts between mgwt and Vaadin}

\subsubsection{SwipeHandler implementation}

\subsection{Vaadin. Developing Custom Vaadin Widgets Using RPC Interfaces and MVP}

\subsubsection{Importance of clear component implementation and fine-grained communication.}

\subsubsection{Widget RPC architecture and application in project}

\subsubsection{Example of appslauncher}

\subsection{Apps}

\subsection{Developing Magnolia 5.0 App Module}

\subsection{Messages App via Server Push Mechanism}

\subsection{Workbench and Pages Editor}

\section{Testing Server Side with JUnit and Mockito}

\pagebreak