/section {misc}

/subsection {so-cms-framework-comparison}
Are both completely different concepts? Or is their an overlap in their meaning?

A web (application) framework is a lower level, generic toolkit for the development of web applications. That could be
any type of system managing and processing data while exposing it's data and services to human users(via web browsers
and other interactive clients) as well as machines via the http protocol.

A CMS is one type of such applications: a system to manage content shown in webiste. Usually/historically, this mainly
means managing (pieces of) text of "pages" shown in a website, and useres that have different levels of access to
manage this content. That's where the C and the M come from.

With a CMS, you can manage web content. With a Web framework, you build web applications.

Would it be correct to say that a Web Framework is used for the creation of a front-end, while a CMS is used for the
back-end?

No. It would be correct to say that a web framework can be used to create a CMS. Both contain parts that work on the
backend as well as on the front end. Often, a CMS is based on a web framework - sometimes CMS developers build there own
web framework, and sometimes they even expose the API of this framework, so a developer can create extensions to the CMS
in a way as if he would develop an application with a web framework. Drupal really does this, so you can create real web
applications based on the integrated framework - with the upside that they will also be easily to integrate into the
CMS. But that(exposing the API of a web framework) is no necessary criteria for being called a CMS.

If yes, then should the Web Framework use the same technology as the CMS? For example could Ruby on Rails be used in
combination with Drupal? Or doesn't that make any sense at all?

It's be possible to combine two existing systems build with these two, (e.g. because you want to show some data in a web
site managed by drupal, that already exists in a Rails-based system). But as Drupal also provides you some of the genric
functionality of it's underlying web framework, it might not be necessary. You would have to manage and learn two very
different systems and handle all the problems with there interoperation. So, I'd try to build a website with only one of
these if possible and only combine them if theres a good reason to.