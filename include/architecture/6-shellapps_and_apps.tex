\section{It Is All About Apps}
\label{section_apps}
Apps. Magnolia 5 introduces applications of simply \emph{apps} to content
management. An \emph{app} stands for the configurable, pluggable unit that
encapsulates some certain functionality that typically operates over the data
stored in repository.
There are two types of apps in Magnolia 5.0 - so called shell apps and simple
ordinary apps. 

\subsection{Shell Apps.} There are three and only three shell apps that are
responsible for administration functions of the system. Those are called
\emph{AppLauncher}, \emph{Pulse} and \emph{Favorites}. \emph{AppLauncher} is a
dashboard with app icons allowing for starting apps and navigating between them.
\emph{Pulse} is an area that contains and receives the notifications and messages for a
user. Finally, \emph{Favorites} aggregates links and bookmarks allowing for better
customization of the system.

\subsection{Apps.} An app is \emph{a tool with a very narrowly focused interface
enabling CMS users to work on one set of closely related tasks or one specific
set of data. An app does not necessarily work on a single, physical data set
(e.g. the pages of a site), but may cover multiple physical data sets required
to solve the task it covers.} \cite{maui_apps}. It is worth mentioning that in
case of Magnolia 5.0 app is not an application per se but rather a concept of a
user interface metaphor.

\subsection{App Framework.}
The \emph{App Framework} is a name for Magnolia 5 functionality
that deals with apps. It also controls app lifecycle events such as starting and
stopping and bringing the app into focus \cite{wiki_app_framework}.

\subsection{App Configuration}
The primary configuation of the apps is done by means of the
\texttt{AppDescriptor} class. \texttt{AppDescriptor} is a simple Plain Old Java
Object (POJO) that holds the meta information like the app name, label and its
icon. A descriptor also carries the name of a class that actually implements the
app and a list of \emph{sub-app} descriptors. A \emph{sub-app} is an entity that
implements the part of an app functionality. The \texttt{SubAppDescriptor} class
resembles an analogous for the app except for the fact the sub-app is a terminal
structure, so it has no child descriptors. 

In order to provide a higher level configuration mechanism for apps it is
possible to describe App descriptors in JCR inside the \emph{config} workspace.
As the Magnolia CMS web application starts up the descriptors are mapped to Java
classes by means of the Node2Bean mechanism:

\begin{figure}[H] \centering \includegraphics[width=\textwidth]{app_configuration.jpg}
	\caption{Registration of Apps.}
	\label{fig:app_registry}
\end{figure}

An instance of the \texttt{ConfiguredAppDescriptorManager} obsreves the changes 
in \emph{config} workspace and translates the JCR nodes that stand for the app descriptors to 
the objects of type \texttt{AppDescriptor}.

The scanned descriptors are then placed into the \texttt{AppDescriptorRegistry}
which fires the corresponding events for registration, deregistration or update.
The main receiver of such kind of events is an implementor of the so called
\texttt{AppLayoutManager} class. The purpose of this class is to maintain the
order of registered apps and to display them properly in the user interface (in
\emph{AppLauncher} shell-app) and to make them available to the navigation
system (the app controllers).

\begin{figure}[H] \centering \includegraphics[width=\textwidth]{app_controller.png}
	\caption{App Context Management}
	\label{fig:app_context}
\end{figure}

\subsection{App Controllers and App Context.} The cornerstones of the \emph{App Framework} are the two
controller objects: \texttt{AppController} and \texttt{ShellAppController}.
These two objects play a similar role as \texttt{ActivityManager} in the original 
Places framework of GWT and apps are the analogues of \texttt{Activity}.
Controllers are subscribed to \texttt{LocationChangeEvent}'s. Based on the
incoming \texttt{Location}s either of two controllers retreives an app context.
\texttt{AppContext} is the class that holds the state of an app that is up and
running.

\begin{lstlisting}
public interface AppContext 
{
    void openSubApp(Location location);
    void start(Location location);
    void stop();
    void onLocationUpdate(Location newLocation);
    
    AppDescriptor getAppDescriptor();
    App getApp();
    
    View getView();
    String getName();    
    String mayStop();
}
\end{lstlisting}

As it is visible from the aforementioned fragment of an interface -
\texttt{AppContext} is able to react on \texttt{Location} changes. The
implementor of an interfaces typicaly conducts it by starting the app or one of
its sub-apps. An \texttt{AppContext} allocates the view component that would
host a new app (typically it is a tab-sheet component). It is also responsible
for providing all the neccessary paramteres to a newly created app or sub-app,
like dependency injector, reference to the context itself (so that the app could
indirectly communicate to the \emph{App Framework}) and location.

As the \texttt{start(\ldots)} method is called an \texttt{AppContext}
instantiates the sub-class of the \texttt{App} interface declared in the app
descriptor.

The \texttt{App} sub-class is basically a \emph{presenter} for a concrete app
implementation. We will cover the examples of app development in the
implementation chapter.