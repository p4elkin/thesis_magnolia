\chapter{Introduction}

World Wide Web (www) plays a significant role in all aspects of our life. It is
a great source of information, key point in business and an easy way for
communication. The building blocks of the WWW are the websites. But what does it
take to make a feature rich, efficient website that is optimized for web search?
Several ways are possible. One obvious option would be to hire a team including
the experienced web-developers, designers and Search Engine Optimization (SEO)
experts. This approach usually allows to achieve the desired result with low
amount of risk. However, the price and development time could also be very high.

The other option is to use a Content Management System (CMS). CMS is a program
or a set of programs used for building websites. Normally CMS is controlled
through some kind of user interface that allows to handle most of the typical
yet crucial tasks out of the box. The main functions of almost every CMS is the
association of content and data with some kind of a rendering mechanism. As a
result the project team can concentrate mostly on business logic and the actual
content while it's presentation can be successfully covered by, for instance, a
CMS template kit. Other tasks that CMS can handle include storing and sharing
data between users, access and user role management, Web pages design,
deployment etc. There are CMS's that could be a great foundation for almost any
fundamental type of web project in various areas like e-commerce, blogs or
social networks. In some cases CMS could let the developer complete the project
without any developer or designer skills.

The biggest problems that could be experienced while using a CMS are usually
related to either steep learning curve or to limitations of the administration
tools. It is especially common in the area of enterprise-oriented Content
Management Systems. The vendor of a powerful CMS has to find a balance between
clear, convenient and intuitive UI and the richness of the functionality in
order to provide customers with full control over their websites. It is also
important that the administration program supports customization.

Magnolia CMS is one of the leading open source Content Management Systems. The
main focus of Magnolia is medium to large enterprise projects. Magnolia CMS is
written in the Java programming language and incorporates various best of breed
Java technologies: data management implemented on top of Jackrabbit (Java
Content Repository implementation), core Application Programming Interface (API)
architecture based on Google Guice dependency injection framework, distribution
and building is handled by Apache Maven.

AdminCentral is one of the most important Magnolia modules. It is the
administration tool of the CMS. The module allows to manipulate the data in JCR
repositories, to run queries and to edit pages with a visual page editor.
The thesis under consideration describes the development of the fifth generation
of Magnolia CMS. The most significant innovation of the new CMS version is the
AdminCentral module developed on top of the Vaadin framework. The main target of
the project is to make the process of administration highly flexible,
customizable and simple.

The cornerstone concept of Magnolia 5.0 is the modular structure based on the
application metaphor. It provides a clear and intuitive understanding of the
system administration principles. This concept also allows the customers and
community contributors to build their own application modules based on simple
API's and to integrate those into the Magnolia admin interface with almost no
effort.

The first chapter of the current text is devoted to the more detailed CMS world
overview. The definition and goals of CMS's will be stated. We will examine the
main types of the existing Content Management Systems, their common useful
features and typical pitfalls. A special emphasis will be devoted to the
position of the Magnolia CMS.

In the following chapter summarizes the main problems that the project under
consideration (Magnolia 5.0) tries to solve. For instance, we will study why it
is improtant that a CMS has a decent support for mobile platforms, what it takes
to create such CMS that is extensive and convenient for user.

Then we will proceed to the observation of the technology stack used in the
considered project (Chapter 4). Even though the range of the used libraries,
frameworks and API's is quite vast - we will examine the most important of them.
For example, the main features of Vaadin, GWT, JQuery and Magnolia CMS itself
will be discussed.

In chapters 5 and 6 we will concentrate on the project's architecture and
implementation. The high-level architecture outlines will be provided in the
chapter 5: the used frameworks role distribution and the relations between them.
The main patterns and concepts of the system that play the most significant
roles will be studied separately. Additional attention for instance will be
devoted the application framework which is the crux or of the project's
architecture.

In chapter 6 we will focus on the implementation details. In that scope we will
study the data binding between Magnolia CMS and the Vaadin framework, some
client-side peculiarities and a sample application development with Magnolia CMS
API.

Finally, we will discuss the the potential flaws and possibilities for the
project's improvement especially by means of the latest version the Vaadin
framework.

Key words: Magnolia CMS, Vaadin, GWT, JQuery, J2EE.
\pagebreak
