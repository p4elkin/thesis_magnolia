Java Content Repository (JCR) is the core underlying technology for storing and
managing data in Magnolia CMS. JCR is a standard programming interface for
communication with content repositories. JCR was first established in Java
Specification Request 170 (JSR-170). Current JCR version 2.0 was finalized in
JSR-283. JCR is able to handle both structured and unstructured content
specified within hierarchical storage. Magnolia CMS uses Apache Jackrabbit
reference implementation of JCR and within the scope of current paper adheres to
the Java Content Repository standard.

\paragraph{Content Repository} stands for a high-level data management system
allowing for various operations to be performed over the stored content. These
are the most essential features of a Content Repository:

\begin{itemize}
  \item Aforementioned ability to store either structured content (e.g. Extensible Mark-up Language (XML)
  files, page templates etc) or unstructured (e.g. Binary Large OBjects (BLOB)).
  \item Referential integrity: support for primary and foreign keys and handling
  the violation of them.
  \item Querying data. The typical language to acces content repository is XML Path Language (XPath) \cite{xpath} 
  XPath. However, Structured Query Language (SQL) dialects usually can also be used.
\end{itemize}

A Magnolia CMS instance works with a single repository (called
\texttt{magnolia}). Semantic separation between datasets is provided with
\emph{workspaces}. Workspace is a tree-like structures with a single root.
The leaves of such a tree are called \emph{properties}. \emph{Properties} carry the all the
actual content of the repository.

From the perspective of this thesis, the most important workspaces of the Magnolia CMS are:
\begin{itemize}
  \item \emph{website} Stores the pages, paragraphs and most of the content  for the websites.
  \item \emph{config} The configuration settings.
  \item \emph{DAM} Workspace for Dynamic Asset Management.
  \item \emph{users} Stores all the types of user accounts (administrative, system public etc).
\end{itemize}

 We will especially concentrate on the \emph{config} workspace as configuration
 will play a significant role in the
foundation for the flexible user interface (see Chapter \ref{architecture}). We will
also touch the \emph{DAM workspace} when we will discuss the Asset Management in
the Chapter \ref{implementation}.
